\documentclass[a4paper,12pt]{article}
\usepackage[utf8]{inputenc}
\usepackage{amsmath}
\usepackage{graphicx}
\usepackage{hyperref}
\usepackage{geometry}
\geometry{left=3cm, right=2.5cm, top=2.5cm, bottom=2.5cm}

\title{\textbf{Proposta de Projeto Final}}
\author{Nome do Estudante}
\date{\today}

\begin{document}

% Capa com imagem
\begin{titlepage}
    \centering
    \includegraphics[width=0.5\textwidth]{istec.jpeg}
    \vspace{1cm}

    {\LARGE \textbf{ISTEC} \par}
    \vspace{1.5cm}

    {\Large \textbf{Projeto Final} \par}
    \vspace{0.5cm}

    \textbf{Goncalo Queirós} \\
    Curso:Desenvolvimento de Software \\

    \textbf{Disciplina} \\
    Desenvolvimento Agil de Software \\

    \vfill

    {\large \today\par}
\end{titlepage}

\newpage
\tableofcontents
\newpage

\section{Introdução}
Este relatório descreve o processo de criação de um repositório GitHub utilizando o modelo GitFlow para gestão de branches, controle de versões e revisão de código. Foram aplicadas práticas recomendadas para gerenciar acessos, configurar revisões obrigatórias e ignorar arquivos desnecessários.

\section{Criação do repositório e configuração do gitflow}


\fbox{
\begin{minipage}{1\linewidth}

c c:repositorios\\
cd projetofinal\\

git init\\

git remote add origin http:/etc\\

git add .\\
git commit -m "primeiro commit"\\

git branch -m main\\
git push -u origin main\\

git flow init\\



git push origin master\\

\end{minipage}
}



\section{Desenvolvimento}
 \fbox{
\begin{minipage}{1\linewidth}

git flow feature start feature1\\
git flow feature finish feature1\\

git flow hotfix start versao1\\
git flow hotfix finish versao1\\

git flow bugfix start versao1\\
git flow bugfix finish versao1\\

git add \texttt{.}gitignore\\
echo "*.doc" \texttt{>>} .gitignore\\
echo "*.docx" \texttt{>>} .gitignore\\

\end{minipage}
}
\\
.\\
\\

\section{Ferramentas}
As ferramentas que foram utilizadas para a conclusão deste relatório foram:

\subsection{Git}
\includegraphics[width=0.4\textwidth]{git.svg.png}
    \vspace{1cm}
    \vspace{0.5cm}
    \begin{itemize}
        \item O Git é um sistema de controle de versão distribuído que permite gerenciar alterações em projetos de forma colaborativa. Ele facilita o trabalho em equipe, mantendo um histórico completo de mudanças e permitindo a criação de ramificações para desenvolvimento isolado. Neste projeto, utilizei o Git para criar o repositório, gerenciar versões e configurei um arquivo .gitignore para excluir arquivos desnecessários.
    \end{itemize}

\subsection{GitHub}
\includegraphics[width=0.2\textwidth]{github.png}
    \vspace{1cm}
    \vspace{0.5cm}
    \begin{itemize}
        \item O GitHub é uma plataforma de hospedagem de código-fonte baseada no Git, que facilita o controle de versão e a colaboração em projetos de software. Ele permite armazenar repositórios, compartilhar código e colaborar com outros desenvolvedores em tempo real. No GitHub, coloquei o arquivo .gitignore para garantir que arquivos desnecessários não fossem versionados e utilizei a plataforma para hospedar os relatórios em PDF, armazenando cinco cópias dos relatórios para facilitar o acesso e a organização do projeto.
    \end{itemize}

\subsection{Overleaf}
\includegraphics[width=0.5\textwidth]{overleaf.jpg}
    \vspace{1cm}
    \vspace{0.5cm}
    \begin{itemize}
        \item Este projeto foi desenvolvido utilizando a plataforma Overleaf, um editor de LaTeX colaborativo e baseado na web. No Overleaf, desenvolvi o documento com o formato e a estrutura necessários para a apresentação do conteúdo, aproveitando todos os recursos que a ferramenta oferece, como a pré-visualização instantânea. Ao finalizar a elaboração do conteúdo, exportei o arquivo como PDF, o qual foi posteriormente colocado no repositório.
    \end{itemize}


\section{Repositorio}

Acesse o repositório: \url{https://github.com/Goncalo842/DAS_Projeto_Final}

\section{Webgrafia}
  \begin{itemize}
  \item Git - https://git-scm.com
  \item GitHub - https://github.com
  \item Overleaf - https://pt.overleaf.com
\end{itemize}





\section{Conclusão}
A utilização do modelo GitFlow permitiu organizar o desenvolvimento de forma clara e eficiente, garantindo que novas funcionalidades fossem integradas com segurança. As configurações de proteção das branches no GitHub e a aplicação de práticas recomendadas fortaleceram o controle de versões e a qualidade do código. Além disso, o uso do Overleaf facilitou a criação deste relatório com uma formatação profissional. Com isso, o projeto alcançou uma gestão mais estruturada e colaborativa.

\end{document}